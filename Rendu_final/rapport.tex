% Options for packages loaded elsewhere
\PassOptionsToPackage{unicode}{hyperref}
\PassOptionsToPackage{hyphens}{url}
\documentclass[
]{article}
\usepackage{xcolor}
\usepackage[margin = 1.8 cm]{geometry}
\usepackage{amsmath,amssymb}
\setcounter{secnumdepth}{-\maxdimen} % remove section numbering
\usepackage{iftex}
\ifPDFTeX
  \usepackage[T1]{fontenc}
  \usepackage[utf8]{inputenc}
  \usepackage{textcomp} % provide euro and other symbols
\else % if luatex or xetex
  \usepackage{unicode-math} % this also loads fontspec
  \defaultfontfeatures{Scale=MatchLowercase}
  \defaultfontfeatures[\rmfamily]{Ligatures=TeX,Scale=1}
\fi
\usepackage{lmodern}
\ifPDFTeX\else
  % xetex/luatex font selection
\fi
% Use upquote if available, for straight quotes in verbatim environments
\IfFileExists{upquote.sty}{\usepackage{upquote}}{}
\IfFileExists{microtype.sty}{% use microtype if available
  \usepackage[]{microtype}
  \UseMicrotypeSet[protrusion]{basicmath} % disable protrusion for tt fonts
}{}
\makeatletter
\@ifundefined{KOMAClassName}{% if non-KOMA class
  \IfFileExists{parskip.sty}{%
    \usepackage{parskip}
  }{% else
    \setlength{\parindent}{0pt}
    \setlength{\parskip}{6pt plus 2pt minus 1pt}}
}{% if KOMA class
  \KOMAoptions{parskip=half}}
\makeatother
\usepackage{color}
\usepackage{fancyvrb}
\newcommand{\VerbBar}{|}
\newcommand{\VERB}{\Verb[commandchars=\\\{\}]}
\DefineVerbatimEnvironment{Highlighting}{Verbatim}{commandchars=\\\{\}}
% Add ',fontsize=\small' for more characters per line
\usepackage{framed}
\definecolor{shadecolor}{RGB}{248,248,248}
\newenvironment{Shaded}{\begin{snugshade}}{\end{snugshade}}
\newcommand{\AlertTok}[1]{\textcolor[rgb]{0.94,0.16,0.16}{#1}}
\newcommand{\AnnotationTok}[1]{\textcolor[rgb]{0.56,0.35,0.01}{\textbf{\textit{#1}}}}
\newcommand{\AttributeTok}[1]{\textcolor[rgb]{0.13,0.29,0.53}{#1}}
\newcommand{\BaseNTok}[1]{\textcolor[rgb]{0.00,0.00,0.81}{#1}}
\newcommand{\BuiltInTok}[1]{#1}
\newcommand{\CharTok}[1]{\textcolor[rgb]{0.31,0.60,0.02}{#1}}
\newcommand{\CommentTok}[1]{\textcolor[rgb]{0.56,0.35,0.01}{\textit{#1}}}
\newcommand{\CommentVarTok}[1]{\textcolor[rgb]{0.56,0.35,0.01}{\textbf{\textit{#1}}}}
\newcommand{\ConstantTok}[1]{\textcolor[rgb]{0.56,0.35,0.01}{#1}}
\newcommand{\ControlFlowTok}[1]{\textcolor[rgb]{0.13,0.29,0.53}{\textbf{#1}}}
\newcommand{\DataTypeTok}[1]{\textcolor[rgb]{0.13,0.29,0.53}{#1}}
\newcommand{\DecValTok}[1]{\textcolor[rgb]{0.00,0.00,0.81}{#1}}
\newcommand{\DocumentationTok}[1]{\textcolor[rgb]{0.56,0.35,0.01}{\textbf{\textit{#1}}}}
\newcommand{\ErrorTok}[1]{\textcolor[rgb]{0.64,0.00,0.00}{\textbf{#1}}}
\newcommand{\ExtensionTok}[1]{#1}
\newcommand{\FloatTok}[1]{\textcolor[rgb]{0.00,0.00,0.81}{#1}}
\newcommand{\FunctionTok}[1]{\textcolor[rgb]{0.13,0.29,0.53}{\textbf{#1}}}
\newcommand{\ImportTok}[1]{#1}
\newcommand{\InformationTok}[1]{\textcolor[rgb]{0.56,0.35,0.01}{\textbf{\textit{#1}}}}
\newcommand{\KeywordTok}[1]{\textcolor[rgb]{0.13,0.29,0.53}{\textbf{#1}}}
\newcommand{\NormalTok}[1]{#1}
\newcommand{\OperatorTok}[1]{\textcolor[rgb]{0.81,0.36,0.00}{\textbf{#1}}}
\newcommand{\OtherTok}[1]{\textcolor[rgb]{0.56,0.35,0.01}{#1}}
\newcommand{\PreprocessorTok}[1]{\textcolor[rgb]{0.56,0.35,0.01}{\textit{#1}}}
\newcommand{\RegionMarkerTok}[1]{#1}
\newcommand{\SpecialCharTok}[1]{\textcolor[rgb]{0.81,0.36,0.00}{\textbf{#1}}}
\newcommand{\SpecialStringTok}[1]{\textcolor[rgb]{0.31,0.60,0.02}{#1}}
\newcommand{\StringTok}[1]{\textcolor[rgb]{0.31,0.60,0.02}{#1}}
\newcommand{\VariableTok}[1]{\textcolor[rgb]{0.00,0.00,0.00}{#1}}
\newcommand{\VerbatimStringTok}[1]{\textcolor[rgb]{0.31,0.60,0.02}{#1}}
\newcommand{\WarningTok}[1]{\textcolor[rgb]{0.56,0.35,0.01}{\textbf{\textit{#1}}}}
\usepackage{longtable,booktabs,array}
\usepackage{calc} % for calculating minipage widths
% Correct order of tables after \paragraph or \subparagraph
\usepackage{etoolbox}
\makeatletter
\patchcmd\longtable{\par}{\if@noskipsec\mbox{}\fi\par}{}{}
\makeatother
% Allow footnotes in longtable head/foot
\IfFileExists{footnotehyper.sty}{\usepackage{footnotehyper}}{\usepackage{footnote}}
\makesavenoteenv{longtable}
\usepackage{graphicx}
\makeatletter
\newsavebox\pandoc@box
\newcommand*\pandocbounded[1]{% scales image to fit in text height/width
  \sbox\pandoc@box{#1}%
  \Gscale@div\@tempa{\textheight}{\dimexpr\ht\pandoc@box+\dp\pandoc@box\relax}%
  \Gscale@div\@tempb{\linewidth}{\wd\pandoc@box}%
  \ifdim\@tempb\p@<\@tempa\p@\let\@tempa\@tempb\fi% select the smaller of both
  \ifdim\@tempa\p@<\p@\scalebox{\@tempa}{\usebox\pandoc@box}%
  \else\usebox{\pandoc@box}%
  \fi%
}
% Set default figure placement to htbp
\def\fps@figure{htbp}
\makeatother
\setlength{\emergencystretch}{3em} % prevent overfull lines
\providecommand{\tightlist}{%
  \setlength{\itemsep}{0pt}\setlength{\parskip}{0pt}}
\usepackage{mdframed}
\usepackage{bookmark}
\IfFileExists{xurl.sty}{\usepackage{xurl}}{} % add URL line breaks if available
\urlstyle{same}
\hypersetup{
  pdftitle={Rapport de projet tutoré},
  pdfauthor={Matthias MAZET, Enzo VIGNAUD, Kento OKADO, Léonie BREUZA},
  hidelinks,
  pdfcreator={LaTeX via pandoc}}

\title{\textbf{\emph{Rapport de projet tutoré}}}
\author{Matthias MAZET, Enzo VIGNAUD, Kento OKADO, Léonie BREUZA}
\date{2025-03-24}

\begin{document}
\maketitle

{
\setcounter{tocdepth}{2}
\tableofcontents
}
\newpage

\begin{center}\rule{0.5\linewidth}{0.5pt}\end{center}

\begin{itemize}
\tightlist
\item
  mettre les bouts de codes en annexe ?
\end{itemize}

\begin{center}\rule{0.5\linewidth}{0.5pt}\end{center}

\section{\texorpdfstring{\textbf{Context}}{Context}}\label{context}

Les greffes de reins possèdent encore aujourd'hui une certaine
probabilité de rejet du greffon par le patient. Cette probabilité
augmente fortement (\textgreater80\%) lorsque ce dernier souffre
d'hyperimmunité, c'est-à-dire que son système immunitaire produit une
quantité excessive d'anticorps, et notamment d'anticorps anti-HLA
responsables du rejet de greffon.\\
Actuellement, les patients atteint d'hyperimmunité suive un protocole
d'immunoabsorption visant à réduire plus ou moins durablement la
quantité d'anticorps anti-HLA dans leur corps, et ainsi augmenté le taux
de réussite de la greffe. En plus d'être lourd a supporté (deux semaines
de prélèvements et traitements sanguin), ce protocole ne garanti pas le
succès de l'opération même si le patient l'achève entièrement. Il semble
donc crucial de pouvoir anticiper les effets des séances
d'immunoabsorption sur chaque patient afin de lui
\textbf{prescrire/suggérer} un protocole adapté et \textbf{allégé}.\\
À l'aide d'un indicateur de la quantité d'anticorps anti-HLA, le Mean
Fluoresence Intensity (MFI), cette étude a donc cherché à modéliser
l'évolution de cette quantité au cours du protocole d'immunoabsorption.
Notamment, \textbf{nous avons (pas génial l'emploi du ``nous'')} essayé
d'anticiper le moment où un patient descendrait en dessous d'un seuil
déterminant de MFI permettant de décider ou non si les chances de succès
d'une greffe serait suffisantes.

\begin{center}\rule{0.5\linewidth}{0.5pt}\end{center}

\begin{itemize}
\tightlist
\item
  Plus de détails sur le protocole (durée, nb de séances, etc.) ?
\item
  Énoncer les différents acteurs (Caroline BAZZOLI, Céline DIARD, Johan
  NOBLE, nous) ?
\item
  Plus de détails sur les anticorps anti-HLA ? Les MFI ?
\item
  Donner la norme de MFI pour Grenoble ? Au niveau national ?
\item
  Autres informations oubliées ?
\end{itemize}

\begin{center}\rule{0.5\linewidth}{0.5pt}\end{center}

\section{\texorpdfstring{\textbf{Méthodes}}{Méthodes}}\label{muxe9thodes}

\subsubsection{\texorpdfstring{\emph{Données}}{Données}}\label{donnuxe9es}

Nous avons travaillé à partir de données longitudinales prélevées sur
dix patients hyperimmunisés en attente d'une greffe de rein à Grenoble.
Chaque patient présente vingt relevés espacés sur deux semaines, soit
deux relevés par jour/séance avec une pause le week-end (généralement).

À partir du fichier original, nous avons sélectionné \textbf{n}
variables, dont certaines que nous avons construites à partir de
variables déjà présentes :

\begin{itemize}
\tightlist
\item
  la variable cible MFI. Le nombre d'anticorps étant trop élevé, il a
  été décidé de d'abord travailler à partir de grande classes, c'est
  pourquoi les MFI sont séparés en deux variables quantitatives
  continues \texttt{MFI\ Classe\ I} et \texttt{MFI\ Classe\ II}.
\item
  les covariables, résumées dans le tableau suivant :
\end{itemize}

\begin{longtable}[]{@{}
  >{\raggedleft\arraybackslash}p{(\linewidth - 4\tabcolsep) * \real{0.2609}}
  >{\raggedleft\arraybackslash}p{(\linewidth - 4\tabcolsep) * \real{0.4783}}
  >{\raggedleft\arraybackslash}p{(\linewidth - 4\tabcolsep) * \real{0.2609}}@{}}
\toprule\noalign{}
\begin{minipage}[b]{\linewidth}\raggedleft
\textbf{Nom}
\end{minipage} & \begin{minipage}[b]{\linewidth}\raggedleft
\textbf{Description}
\end{minipage} & \begin{minipage}[b]{\linewidth}\raggedleft
\textbf{Type/Modalités}
\end{minipage} \\
\midrule\noalign{}
\endhead
\bottomrule\noalign{}
\endlastfoot
\texttt{Durée\ de\ la\ séance\ (min)} & Durée de la séance \textbf{=
durée du tratiement du plasma ?} (en min) & Continue \\
\texttt{Volume\ plasma\ traité} & Volume de plasma traité lors de la
séance d'immunoabsorption en question. (en mL) & Continue \\
\texttt{Poids\ (kg)} & Poids du patient lors de la séance. (en kg) &
Continue \\
\texttt{Taille\ (cm)} & Taille du patient lors de la séance. (en cm) &
Continue \\
\texttt{Sexe} & Sexe du patient. & Binaire (H/F) \\
\texttt{Grossesses} & Nombre de grossesses du patient antérieures au
protocole. & Discrète (0 à 4 + NC pour les hommes) \\
\texttt{Greffe\ antérieure} & Nombre de greffes \textbf{(de rein ?)}
effectué sur le patient avant le protocole. & Discrète (0 ; 1 ; 2) \\
\textbf{AUTRES ?} & & \\
& & \\
\texttt{Temps}\(^*\) & Écoulement du temps au fil du protocole, du
premier au dernier prélèvement. (en min) & Continue \\
\texttt{delta\ MFI}\(^*\) & Variation de la quantité de MFI entre le
début et la fin d'une même séance. & Continue \\
\end{longtable}

\emph{\(^*\) : covariables construites.}

\begin{Shaded}
\begin{Highlighting}[]
\CommentTok{\# Détails + code sur la gestion des valeurs manquantes et autre nettoyage de données}
\end{Highlighting}
\end{Shaded}

À partir des variables \texttt{Date\ séance}, \texttt{Heure\ début} et
\texttt{Heure\ fin}, nous avons aussi construit une temporalité continue
\texttt{Temps} commençant à \(t_0\) = début de la première séance (min)
et finissant à \(t_{20}\) = fin de la dernière séance (min). Cette
variable permet de savoir rapidement la durée du protocole chez un
patient ainsi que de construire des graphiques de statistiques
descriptives correctes.

\begin{Shaded}
\begin{Highlighting}[]
\CommentTok{\# Plus de détails (?) + code}
\end{Highlighting}
\end{Shaded}

\subsubsection{\texorpdfstring{\emph{Protocole}}{Protocole}}\label{protocole}

Actuellement, le protocole d'immunoabsorption se réparti en dix séances
de désimmunisation réparties sur 2 semaines avec une pause le week-end.
Chaque séance vise à faire baisser le taux d'anticorps anti-HLA chez un
patient en prélevant son plasma et en le traitant à l'aide d'une colonne
de désimmunisation. Ce taux est contrôlé via une quantité de MFI et le
but à la fin des dix séances est de le faire descendre en dessous de 3
000, la norme pour Grenoble. Il est à noté que la norme au niveau
national est de 2 000 MFI.

Afin de suivre l'évolution de la quantité de MFI chez un patient, deux
relevés sont réalisés à chaque séance, un juste avant le prélèvement et
traitement du plasma et un autre à la fin de la séance.

\subsubsection{\texorpdfstring{\emph{Analyses
statistiques}}{Analyses statistiques}}\label{analyses-statistiques}

Outils utilisés : R et Monolix.

Stat univariées/bivariées :

\begin{itemize}
\tightlist
\item
  décrire lesquelles faire et pq -\textgreater{} MFI en fonction du
  temps pour voir leur évolution, delta MFI en fonction du temps, durée
  de la séance en fonction du patient, etc.
\end{itemize}

Descriptions des modèles statistiques (contexte d'application + formule)
:

\begin{itemize}
\tightlist
\item
  modèles K-PD :
\end{itemize}

\emph{Modèle pharmacocinétiques (PK)} : décrire et prédire l'évolution
de la concentration du médicament en fonction du temps.

\emph{Modèle pharmacodynamiques (PD)} : décrire et prédire la
réponse/l'effet observé(e) lors d'un événement pharmacologique,
i.e.~lors d'une interaction entre notre organisme et un médicament.

\emph{Modèle PK-PD} : décrire et prédire l'interaction entre une dose de
médicament et l'organisme au cours du temps. Permet de trouver un niveau
de réponse allant de ``efficace'' à ``toxique''.

Dans notre cas : pas de ``dose'' à proprement parler, donc pas possible
d'utiliser directement un modèle PK. Alternative =\textgreater{}
\emph{modèles K-PD}. Ces modèles permettent de simuler/ajuster une dose
de médicament lorsqu'elle n'est pas précisément définie afin d'obtenir
malgré tout un modèle détaillant l'interaction médicament/organisme,
plutôt que simplement un modèle PD.

\begin{itemize}
\tightlist
\item
  modèles Tumor growth inhibition (TGI) :
\end{itemize}

Modèles permettant de décrire et prédire l'évolution d'une tumeur au
cours du temps à partir d'une taille initiale, ainsi que l'effet qu'un
traitement peut avoir sur cette dite évolution.

Ces modèles nous semblaient intéressant vis-à-vis de la notion de
``taille initiale'' qu'ils mettent en jeu.

Simulation des protocoles : méthodes utilisées + critères de validation.

\begin{center}\rule{0.5\linewidth}{0.5pt}\end{center}

\begin{itemize}
\tightlist
\item
  Détailler la construction de la variable \texttt{delta\ MFI} ?
\item
  Plus de détails pour la partie ``Protocole'' ?
\item
  Mettre des graphiques/tableaux dans la partie ``Analyses
  statistiques'' ou les gardés pour la partie \textbf{``Résultats''} ?
\item
  Les modèles PK peuvent servir à décrire d'autres concentration
  (globules blancs, anti-corps, etc.) ou seulement celle du médicament
  administré ?
\end{itemize}

\begin{center}\rule{0.5\linewidth}{0.5pt}\end{center}

\section{\texorpdfstring{\textbf{Résultats}}{Résultats}}\label{ruxe9sultats}

\subsubsection{\texorpdfstring{\emph{Statistiques
univariées/bivariées}}{Statistiques univariées/bivariées}}\label{statistiques-univariuxe9esbivariuxe9es}

\begin{itemize}
\tightlist
\item
  Profils des patients : taille, âge, sexe, grossesses, etc.
\item
  Boxplots des durée de séance par patient, du volume de plasma traité
  par patient (Cf présentation intermédiaire).
\item
  visualisation du nb de patients sous les 3000 MFI à chaque séance (Cf
  présentation intermédiaire =\textgreater{} graphique à améliorer).
\end{itemize}

\begin{Shaded}
\begin{Highlighting}[]
\CommentTok{\# Graphiques pertinents}
\end{Highlighting}
\end{Shaded}

\subsubsection{\texorpdfstring{\emph{Modèles}}{Modèles}}\label{moduxe8les}

Résultats des modèles (graphiques + indices de qualité/comparaison) :

\begin{itemize}
\tightlist
\item
  modèles K-PD.
\item
  modèles TGI.
\end{itemize}

Meilleur modèle : lequel et pourquoi.

\section{\texorpdfstring{\textbf{Discussion/Conclusion}}{Discussion/Conclusion}}\label{discussionconclusion}

Objectif de l'étude : modéliser l'évolution de la quantité de MFI au
cours du protocole afin d'anticiper le moment où une greffe devient
possible =\textgreater{} réduire la charge du protocole pour les
patients.

Principaux résultats :

\begin{itemize}
\tightlist
\item
  Quelle(s) covariable(s) influence(nt) l'évolution du taux de MFI ?
\item
  Meilleur modèle : rappel de ses paramètres d'entrée + fiabilité
\end{itemize}

Limites de l'étude (hypothèses, a priori\ldots) :

\begin{itemize}
\tightlist
\item
  Pas assez de patients/données ?
\item
  Compréhension du système immunitaire encore trop limité ?
\end{itemize}

Ouverture : qu'est-ce qu'il pourrait être fait par la suite ?

\begin{itemize}
\tightlist
\item
  Améliorer la précision des prélèvements ?
\end{itemize}

\section{\texorpdfstring{\textbf{Impact Environnemental et
Sociétal}}{Impact Environnemental et Sociétal}}\label{impact-environnemental-et-sociuxe9tal}

Cf site et pdf sur Moodle.

\subsubsection{\texorpdfstring{\emph{Impact environnemental
personnel}}{Impact environnemental personnel}}\label{impact-environnemental-personnel}

\begin{itemize}
\tightlist
\item
  Trajets domicile-travail et autres déplacements.
\item
  Consommation des équipements utilisés (ordinateurs fixes ou portables,
  temps d'utilisation serveur, etc.).
\item
  Autres impacts ?
\end{itemize}

\subsubsection{\texorpdfstring{\emph{Impact global du
projet}}{Impact global du projet}}\label{impact-global-du-projet}

\subparagraph{Impact environnemental}\label{impact-environnemental}

\subparagraph{Impact sociétal}\label{impact-sociuxe9tal}

\subsubsection{\texorpdfstring{\emph{Politique de la structure
d'accueil}}{Politique de la structure d'accueil}}\label{politique-de-la-structure-daccueil}

Pas concerné pour ce projet. (?)

\end{document}
